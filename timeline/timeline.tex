\documentclass[aasms,12pt]{article}
\usepackage{natbib}
\setlength{\bibsep}{0pt plus 0.3ex}
\usepackage{sectsty}
\usepackage{graphicx}
\usepackage{epstopdf}
\usepackage[skip=2pt,font=small]{caption}
\captionsetup{width=\textwidth}


%\usepackage{titlesec}
%
%\titlespacing\section{0pt}{12pt plus 4pt minus 2pt}{5pt plus 2pt minus 2pt}
%\titlespacing\subsection{0pt}{12pt plus 4pt minus 2pt}{0pt plus 2pt minus 2pt}
%\titlespacing\subsubsection{0pt}{12pt plus 4pt minus 2pt}{0pt plus 2pt minus 2pt}


\sectionfont{\normalsize}
\usepackage{fullpage}

%\usepackage[margin=1in]{geometry}

%\citestyle{aa}
\newcommand{\sol}{\ensuremath{\odot}}


\usepackage{fancyhdr}
\pagestyle{fancy}
\fancyhf{} % sets both header and footer to nothing
\renewcommand{\headrulewidth}{0pt}
\rfoot{\footnotesize{Evan Anders, NASA NESSF 2015}}




\begin{document}
\begin{center}
   \large\textbf{Timeline of Graduate Studies}\\
   \vspace{0.4cm}
   \large{Evan Anders}\\
   \normalsize\textit{Department of Astrophysical and Planetary Sciences}\\
   \normalsize\textit{University of Colorado at Boulder}\\
\end{center}

\paragraph{August 2014:} Begin graduate studies.  Begin graduate coursework;
	assume position of Teaching Assistant for 2014-2015 academic year
\vspace{-0.4cm}
\paragraph{May 2015:} Begin work as a graduate Research Assistant with
	Benjamin P. Brown.

\vspace{-0.4cm}
\paragraph{September 2015:} Start of proposed funding from NESSF.

\vspace{-0.4cm}
\paragraph{Fall 2015:} Build initial dynamo models with and without tachocline.

\vspace{-0.4cm}
\paragraph{January 2016:} (1) Complete first departmental qualifier for PhD candidacy,
``COMPS I'', a comprehensive test over five core courses: Atomic \& Molecular Processes,
Mathematical Methods, Fluid Mechanics, Observations \& Statistics, and Radiative
Processes.  (2) Begin assessing the tachocline's role in setting cycle periods.

\vspace{-0.4cm}
\paragraph{May 2016:} Complete graduate courses.  Further time in the department
will be devoted entirely to research.

\vspace{-0.4cm}
\paragraph{Fall 2016:} Write and publish a peer-reviewed article on findings
regarding the tachocline's role.

\vspace{-0.4cm}
\paragraph{December 2016:} Complete second departmental qualifier for PhD candidacy,
``COMPS II'', a defense of completed research up until this point.  If successful,
move on to PhD candidacy.

\vspace{-0.4cm}
\paragraph{Spring 2017:} Build dynamo models which include solar photosphere.
Begin development of post-processing techniques to create simulated observables.

\vspace{-0.4cm}
\paragraph{Summer 2017:} Create simulated observables which parallel data taken
by the Solar Dynamics Observatory (SDO) and the future Solar Orbiter.

\vspace{-0.4cm}
\paragraph{Fall 2017-Spring 2018:} Analyze simulated observables, comparing them to current
SDO data.  Use simulated observables to predict phenomenon that the Solar Orbiter
will observe.

\vspace{-0.4cm}
\paragraph{Summer 2018:} Write and publish a peer-reviewed article on findings
and predictions derived from simulated observables.

\vspace{-0.4cm}
\paragraph{August 2018:} Completion of proposed funding from NESSF.

\vspace{-0.4cm}
\paragraph{May 2019:} Anticipated graduation date

\end{document}
