\documentclass[aasms,12pt]{article}
\usepackage{natbib}
\setlength{\bibsep}{0pt plus 0.3ex}
\usepackage{sectsty}
\usepackage{graphicx}
\usepackage[skip=2pt,font=small]{caption}
\captionsetup{width=\textwidth}


\sectionfont{\normalsize}
\usepackage{fullpage}

%\usepackage[margin=1in]{geometry}

%\citestyle{aa}
\newcommand{\sol}{\ensuremath{\odot}}


\usepackage{fancyhdr}
\pagestyle{fancy}
\fancyhf{} % sets both header and footer to nothing
\renewcommand{\headrulewidth}{0pt}
\cfoot{\thepage}
\rfoot{\footnotesize{Evan Anders, NASA NESSF 2015}}




\begin{document}
\begin{center}
   \Large\textbf{Determining the role of the tachocline in solar
	dynamo creation; Modeling the surface appearance of solar magnetic
	fields (Need to cogently combine these ideas)}\\
   \vspace{0.4cm}
   \large{Evan Anders}\\
   \vspace{0.4cm}
   \normalsize\textit{Advisor: Benjamin Brown}\\
   \normalsize\textit{LASP, University of Colorado at Boulder}\\
\end{center}



\abstract{The Sun is magnetically active due to a magnetic dynamo which is
	powered by the solar convective zone.  Numerical simulations have
	offered great insight into the nature of the magnetic fields within
	the Sun's convective zone.  However, generally these simulations
	have been done for too large of rotation rates and too large of
	characteristic convective motions.  Here I propose to create a
	suite of 3-D numerical simulations of the solar convective zone
	with accurate solar rotation and velocity profiles.  My primary
	goal is to determine the role of the solar tachocline in producing
	and sustaining the solar magnetic dynamo.  Furthermore, a more
	detailed set of simulations which include large-scale details of
	the solar photosphere will be created and post-processed in order
	to directly compare simulation data with data such as that gathered
	by spacecraft such as the Solar Dynamics Observatory.}



\section{Intro}
The Sun is a magneticly active star.  Its magnetism arises from an 
organized dynamo which is seated in the turbulent plasma
motions in the solar convection zone, which constitutes roughly the outer 30\%
of the solar radius.  In the presence of
convective motions, magnetic fields and solar rotation couple to produce global
wreaths of magnetism which drive the Sun's 22-year cycle of magnetic activity.
This activity manifests itself in the collection of phenomena which is generally
referred to as solar activity, including magnetic storms and coronoal mass
ejections.  Such activity propagates towards Earth, threatening to disrupt power
grids and aircraft operations and to endanger astronauts and satellites.  It is
clear that the Sun's magnetism affects our increasingly technological society,
and understanding the nature of the dynamo that drives solar magnetism is of
paramount importance.

While current models lack the power to accurately 
predict the stellar magnetic environment 
over long time scales, our understanding of the workings of the solar dynamo 
have
vastly improved over the past decade.  Helioseismology, which probes the Sun's
radial structure and sub-surface convective flows using acoustic oscillations,
has revealed that the solar differential rotation profile extends through the
bulk of the convective zone with two distinct shear regions.  A near-surface
shear layer occupies roughly the outer 5\% of the Sun and a secondary shear
layer, called the tachocline, resides at the base of the convective zone
and separates the differentially rotating convective zone from the uniformly
rotating radiative zone.  Traditionally, it has been believed that the
tachocline generates the magnetic dynamo.  In such a model,
magnetic fields generated
in the convective zone are transported inward to the tachocline, where
magnetic wreaths are built before buoyantly 
rising to the solar surface and manifesting as sunspots.

The current theory of the processes that govern the solar dynamo has been
largely informed by large-scale 3-D numerical simulations.  Such simulations
have proven that global magnetic structures can be generated and persist 
even in the turbulence of the solar convective zone.  However, most simulations
of solar convection are truly simulations of Sun-like stars rather than the Sun
itself.  Such simulations model starts with greater rotation rates and greater
convective flow velocities than our sun.  
It has recently been discovered
that these simulations may not perfectly reflect the physics that happens
in the Sun as the amplitudes of large scale convective motions in the sun are
likely much smaller than anticipated and assumed in past simulations
(\citealt{lord2014}).  
Through lowering the stellar rotation rate to that of the solar rotation rate
(roughly 25 days around the equator and 35 days around the poles)
and decreasing the amplitudes of convective motions within modeled solar
convection zones, it is now possible to gain an understanding of the physics at
work within the Sun itself through simulations, rather than just sun-like
stars.

In the context of accepted dynamo theory, an unsettling result has arisen
in recent simulations of sun-like stars in that there is a disagreement
regarding the necessity of the tachocline
in generating the solar magnetic dynamo.  Stable solar dynamos have been
created in 3-D numerical simulations both with 
(\citealt{ghizaru2010, racine2011})
and without (\citealt{brown2011, nelson2013}) the presence of a tachocline. This
begs the fundamental question: is the tachocline a necessary ingredient in the
production of the solar toroidal magnetic field?  Furthermore, is a tachocline
necessary to produce a solar dynamo?  Those simulations which produced opposing
results were made with different codes using differing models for subgrid
processes at various stellar rotation rates.  Consequently, it is impossible to
meaningfully compare the results of the simulations directly.  The data required
to determine the necessity and role of the tachocline are missing.

Furthermore, while simulations of stellar magnetic dynamos have proven
irreplaceable in gaining insight to the structure and evolution of 
long-term cycles, they lack applicability to measured data and day-to-day
predictive power.  With the wealth of spacecraft currently gathering data on
the Sun in addition to the numerous missions which are scheduled to launch
within the next decade, now is the time to bridge the gap between theoretical
dynamo simulations and direct observations of solar activity.

\section{Proposed Research: Tachocline}
I propose a series of hydrodynamical simulations of the Sun's convective zone
which test the role of a tachocline in the generation of a solar magnetic 
dynamo.  This suite of simulations will model the majority of the convective
zone, from its base around $R = 0.713R_{\odot}$ up to nearly its outer layer
around $R = 0.97R_{\odot}$.  The primary differences between simulations in this
suite will be whether or not the convective zone is modelled with or without a
tachocline at its base.  While it has been the general assumption that toroidal
magnetic wreaths are generated within the tachocline, this assumption has not 
yet been put to the test.  As a keystone of the theory of solar dynamo creation,
the importance of the tachocline must be understood in order to make any
meaningful predictions of the behavior of solar magnetism.

I will use the widely-known Anelastic Spherical Harmonic (ASH, 
\citealt{clune1999}) spectral solver, which I will have access to through my
advisor, Ben Brown, to create my simulations.  The use of a well-established
code such as ASH will allow me to
efficiently model the solar convective zone in spherical coordinates and gain
and understanding of the physics driving the solar dynamo without having to
worry about generating a fully functional modelling suite.  Naturally, work
of this magnitude requires access to massively parallel computing resources.
As a CU Boulder student, I have access to the school's local supercomputer,
\emph{Janus}.  Additionally, I will work with Ben Brown to acquire CPU time on
state-of-the art supercomputers such as the NSF XSEDE resources and NASA's
\emph{Pleiades}. 





\section{Proposed Research: Observables}
While simulations are extremely beneficial in gaining an understanding of the
physical trends occuring within a system, they often fall short of a direct
connection to observable data.  Thus, I propose a second suite of simulations:
one which spans the entirety of the solar convective zone and reaches out to
the solar photosphere in order to determine the behavior of the magnetic fields
at the observable surface of the Sun.  In efforts to model the true physics of
the Sun as acurately as possible, these simulations will capture supergranular 
scales but will ignore small scale granulation.  This simplification will be
made for two reasons.  First, the small scale motion of granulation is unlikely
to greatly affect the overall motions of the vast solar convective zone (as
they have characteristic length scales roughly two orders of magnitude apart).
Second, the computational load that would be required to resolve granular scales
would push the number of CPU hours required for a project of this magnitude
past that which is feasible to acquire on state-of-the-art machines.

Numerous post-processing techniques would be used on the photospheric data 
given by these simulations in order to create synthetic observables.
Atmospheric and radiative transfer models would be used to transform simulation
data (which is known precisely in terms of fields and velocities) into
observable phenomena (namely, sunspots). 
Presumably, simulations
which accurately capture the physics at work within the solar convective zone
will create patterns similar to those observed at the solar surface.  If
simulated sunspots behave similarly to real sunspots, then we know that our
theory of the dynamo is working properly and we will have taken a step towards
being able to predict solar weather through pattern recognition.

The synthetic observables created by these simulations would be directly
comparable to data from the Solar Dynamics Observatory (SDO) and the Van Allen
Probes.  Those structures and behavioral patterns which arise in the results
of the proposed simulations can offer insights into the workings of the real
solar convective zone and can potentially be used to predict unusual solar
activity in the future.  Furthermore, the scheduled Solar Orbiter Collaboration
(SCO) will be the first spacecraft to study the solar poles in detail.  While
current spacecraft will prove beneficial in drawing ties between theoretical
structure and actual magnetic structures near the solar equator, SCO will offer
an opportunity to understand the results of such simulations along the solar
poles. 

\begin{figure}[t!]
\centering
\includegraphics[width=14cm]{2014_oct_sunspots.jpg}
\caption{SDO Colorized magnetogram images of solar active region AR 12192, taken
	on 10/19/2014, 10/23/2014, and 10/27/2014, respectively.  Simulated
	observables could mimic the behavior of regions such as this and help
	us understand when and why they release solar flares such as the X3.1
	flare released on 10/24/2014.
	\label{AR12192}}
\end{figure}

\section{My Qualifications}
My education has prepared me to think as both a physicist and a computer
scientist when approaching problems and writing code.  In 2012, I worked with
Pacific Northwest National Laboratory's (PNNL) Data Intensive Scientific
Computing group and gained an understanding of the challenges faced in the
creation of large, scientific computations.  In addition to learning the
struggles faced in efficiently creating massively parallelised algorithms,
I learned how to effectively understand and utilize computational tools 
(such as ASH) created by others for my own purposes. Furthermore, I learned
numerous techniques for debugging and optimizing my routines---techniques
which will certainly come in handy while trying to get a simulation base
fully operational and ready to be sent on to a supercomputer.

Furthermore, over the summer of 2013, I participated in the NSF Science
Undergraduate Research Fellowship program at the Laser Interferometer
Gravitational-wave Observatory (LIGO).  During my time at LIGO Hanford, it was
my task to create a computational tool which analyzed LIGO science run data at
specific frequencies and output information about the data at those frequencies
in user-friendly text files.  This experience taught me how to interact with
massive quantities of data, how to organize that data meaningfully in files,
and how to effectively plot and visualize such data.  All of these skills
will be exceptionally useful at all stages in massive 3D simulations.

Coupled with my strong undergraduate education in physics, I am in the process
of learning the fundamentals of fluid mechanics and plasma physics necessary to
understand the processes which govern the motion of the solar convective zone.
As such, by the beginning of the 2015 academic year, I will be well-poised to
tackle these proposed problems. 


\section{Relevance to NASA} 
The proposed work fits perfectly with NASA's 2014 Strategic plan objectives
1.4:
``Understand the Sun and its interactions with Earth and the solar
system, including space weather.''
This work also fits in with one of the three overarching science goals
of the Heliophysics section of NASA's 2014 Science plan: 
``Develop the
knowledge and capability to detect and predict extreme conditions in space to
protect life and society and to safeguard human and robotic explorers beyond
earth.''

In order to predict extreme space conditions caused by the Sun's magnetic
activity, we must have an intricate understanding of the behavior of the Sun's
magnetic dynamo.  While solar dynamo theory has progressed impressively over
recent years, it has progressed with an untested assumption as a cornerstone
and a series of impressive simulations with no tangible connection to
observables.  After recent simulations have disagreed regading the importance
of the tachocline in the production of the solar magnetic dynamo, it is time
to put the assumed importance of the tachocline to the test.  Furthermore,
it is time to connect simulations to the real world.  In order to have a
consistent, predictive theory on the behavior of the solar magnetic field, it
is necessary to connect theoretical calculations---those largely present in
large-scale numerical simulations---with tangible observables.  Only once these
two sources of data are connected will we be able to possibly predict upcoming
solar magnetic behavior.



\nocite{*}
\bibliographystyle{apj}
\begingroup
\renewcommand{\section}[2]{}%
\begin{footnotesize}
\bibliography{biblio}
\end{footnotesize}
\endgroup
\end{document}
