\documentclass[aasms,12pt]{article}
\usepackage{natbib}
\setlength{\bibsep}{0pt plus 0.3ex}
\usepackage{sectsty}

\sectionfont{\normalsize}
\usepackage[margin=1in]{geometry}
%\citestyle{aa}
\newcommand{\sol}{\ensuremath{\odot}}


\usepackage{fancyhdr}
\pagestyle{fancy}
\fancyhf{} % sets both header and footer to nothing
\renewcommand{\headrulewidth}{0pt}
\lfoot{Research Proposal}
\cfoot{\thepage}
\rfoot{Evan Anders, NASA NESSF}




\begin{document}
\begin{center}
   \Large\textbf{Working Title}\\
   \vspace{0.4cm}
   \large{Evan Anders}\\
   \vspace{0.4cm}
   \normalsize\textit{Advisor: Benjamin Brown}\\
   \normalsize\textit{LASP, University of Colorado at Boulder}\\
\end{center}



\abstract{Fill this later}


\section{Outline (to remove later)}
\begin{itemize}
\item ``The Sun is magnetically active, and this affects our technological
	society..."
\item Talk about the history of solar magnetic simulations.  Show the results that
	they have achieved (wreaths of magnetism are maintained by the convective
	zone alone, most are really high rotation rate but that's probably
	wrong now, yadda yadda...). 
\item Talk about how my simulations proposed with enrich that history of
	simulations (e.g. there's been the ``assumption'' almost that the
	toroidal field is made in the tachocline, and this will help to distinguish
	whether or not that's the truth in an actual ``laboratory''.
\item Talk about the tools that will be used in the simulations.  We have ASH.
	Maybe we'll use Dedalus?
\item Talk about how, once this matter is settled, we'll bridge the gap towards
	``observables'' by extending simulations to the solar surface at the
	resolution of supergranulation.  After running simulations that encompass
	``important'' characteristics of the surface (on our length scales),
	we'll do post-processing in order to convert our simulations into
	``observables,'' and see if such observables look like anything that
	(X, Y, Z) sattelites could detect/have detected.  This is a (possible)
	glimpse at what happens underneath the surface of the sun.
\item Mention the NASA strategic/science plans and exactly which points of them
	we're fitting into, here.
\end{itemize}



\section{Intro}
(I'll take this part from Ben's past work tomorrow when I have an ounce of energy).




Sources and stuff below so I can see formatting!


\nocite{*}
\bibliographystyle{apj}
\begingroup
\renewcommand{\section}[2]{}%
\begin{footnotesize}
\bibliography{biblio}
\end{footnotesize}
\endgroup
\end{document}
