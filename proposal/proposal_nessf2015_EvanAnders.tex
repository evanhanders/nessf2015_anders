\documentclass[aasms,12pt]{article}
\usepackage{natbib}
\setlength{\bibsep}{0pt plus 0.3ex}
\usepackage{sectsty}

\sectionfont{\normalsize}
\usepackage[margin=1in]{geometry}
%\citestyle{aa}
\newcommand{\sol}{\ensuremath{\odot}}


\usepackage{fancyhdr}
\pagestyle{fancy}
\fancyhf{} % sets both header and footer to nothing
\renewcommand{\headrulewidth}{0pt}
\cfoot{\thepage}
\rfoot{Evan Anders, NASA NESSF 2015}




\begin{document}
\begin{center}
   \Large\textbf{Working Title}\\
   \vspace{0.4cm}
   \large{Evan Anders}\\
   \vspace{0.4cm}
   \normalsize\textit{Advisor: Benjamin Brown}\\
   \normalsize\textit{LASP, University of Colorado at Boulder}\\
\end{center}



\abstract{Fill this later}



\section{Intro}
The Sun is a magnetic star whose magnetism arises from an organized dynamo. 
Turbulent plasma
motions in the solar convection zone, which constitutes roughly the outer 30\%
of the solar radius, are the seat of the magnetic dynamo.  In the presence of
convective motions, solar rotation and magnetism couple to produce global
wreaths of magnetism which drive the Sun's 22-year cycle of magnetic activity.
This activity manifests itself in the collection of phenomena which is generally
referred to as solar activity, including magnetic storms and coronoal mass
ejections.  Such activity propagates towards Earth, threatening to disrupt power
grids and aircraft operations and to endanger astronauts and satellites.  It is
clear that the Sun's magnetism affects our increasingly technological society,
and understanding the nature of the dynamo that drives solar magnetism is of
paramount importance.

While current models lack the power to predict the stellar magnetic environment 
over long time scales, understanding of the workings of the solar dynamo have
vastly improved over the past decade.  Helioseismology, which probes the Sun's
radial structure and sub-surface convective flows using acoustic oscillations,
has revealed that the solar differential rotation profile extends through the
bulk of the convective zone with two distinct shear regions.  A near-surface
shear layer occupies roughly the outer 5\% of the Sun and a secondary shear
layer, called the tachocline, resides at the base of the convective zone
separating the differentially rotating convective zone from the uniformly
rotating radiative zone.  Traditionally, it has been believed that the
tachocline is the seat of the magnetic dynamo, in that magnetic fields generated
in the convective zone are transported inward to this shear region, building
magnetic wreaths which buoyantly rise to the surface and manifest as sunspots.

Our current theory of the processes that govern the solar dynamo has been
largely informed by large-scale 3-D numerical simulations.  Such simulations
have proven that global magnetic structures can persist and be generated
even in the turbulence of the solar convective zone.  However, most of these
simulations have not truly been simulations of the Sun but rather of solar-type
stars with significantly larger rotation rates and rapid convective motions.  
It has recently been discovered
that these simulations may not accurately reflect the physics that happens
in our sun and the convective motions being assume may be far too large.
Through lowering the stellar rotation rate and convective motions used in
most current simulations of sun-like stars, now is the time to actually catch
a glimpse of the physics at work within the Sun. 

Recent simulations have showed disagreement on the necessity of the tachocline
in the generation of the solar magnetic dynamo.  Stable solar dynamos have been
created in 3-D numerical simulations both with 
(\citealt{ghizaru2010, racine2011})
and without (\citealt{brown2011, nelson2013}) the presence of a tachocline. This
begs the fundamental question: is the tachocline a necessary ingredient in the
production of the solar toroidal magnetic field?  Furthermore, is a tachocline
necessary to produce a solar dynamo?  Those simulations which produced opposing
results were made with different codes using differing models for subgrid
processes at various stellar rotation rates.  Consequently, it is impossible to
meaningfully compare the results of the simulations directly.  The data required
to determine the necessity and role of the tachocline are missing.

\section{Proposed Research: Tachocline}






\section{Proposed Research: Observables}




\section{Relevance to NASA} 


\section{Outline (to remove later)}
\begin{itemize}
\item Talk about how my simulations proposed with enrich that history of
        simulations (e.g. there's been the ``assumption'' almost that the
        toroidal field is made in the tachocline, and this will help to distinguish
        whether or not that's the truth in an actual ``laboratory''.
\item Talk about the tools that will be used in the simulations.  We have ASH.
        Maybe we'll use Dedalus?
\item Talk about how, once this matter is settled, we'll bridge the gap towards
        ``observables'' by extending simulations to the solar surface at the
        resolution of supergranulation.  After running simulations that encompass
        ``important'' characteristics of the surface (on our length scales),
        we'll do post-processing in order to convert our simulations into
        ``observables,'' and see if such observables look like anything that
        (X, Y, Z) sattelites could detect/have detected.  This is a (possible)
        glimpse at what happens underneath the surface of the sun.
\item Mention the NASA strategic/science plans and exactly which points of them
        we're fitting into, here.
\end{itemize}



Sources and stuff below so I can see formatting!


\nocite{*}
\bibliographystyle{apj}
\begingroup
\renewcommand{\section}[2]{}%
\begin{footnotesize}
\bibliography{biblio}
\end{footnotesize}
\endgroup
\end{document}
