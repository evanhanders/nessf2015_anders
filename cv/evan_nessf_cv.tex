\documentclass[11pt, preprint]{aastex}

%%%%%%begin preamble
\usepackage[hmargin=1in, vmargin=1in]{geometry} % Margins
\usepackage{hyperref}
\usepackage{url}
\usepackage{times}
\usepackage{natbib}
\usepackage{graphicx}
\usepackage{amsmath}
\usepackage{amsfonts}
\usepackage{amssymb}
\usepackage{pdfpages}
\usepackage{import}
%\usepackage{fontspec}
%\setmainfont{TimesNewRoman}
%\definetypeface[Serapion][rm][Xserif][Serapion Pro]
%\setupbodyfont[Serapion, 12pt]

%%%
%%%%%% uncomment following 4 lines to adjust title size/shape and
%%%%%% trailing space
%% \usepackage{titling}
%% %\pretitle{\noindent\Large\bfseries}
%% \date{}
%% \setlength{\droptitle}{-1in}
%\posttitle{\\}

\hypersetup{
     colorlinks   = true,
     citecolor    = gray,
     urlcolor      = blue
}

\setcounter{tocdepth}{2}
%% headers
\usepackage{fancyhdr}
\pagestyle{fancy}
\lhead{Brown \& Anders}
\chead{}
\rhead{Title goes here}
\lfoot{}
\cfoot{\thepage}
\rfoot{}

\newcommand{\sol}{\ensuremath{\odot}}
\newcommand{\Dedalus}{\href{http://dedalus-project.org}{Dedalus}}
\newcommand{\Reyn}{\ensuremath{\mathrm{Re}}}
\newcommand{\Rayleigh}{\ensuremath{\mathrm{Ra}}}
\newcommand{\Rossby}{\ensuremath{\mathrm{Ro}}}
\newcommand{\Rmag}{\ensuremath{\mathrm{Rm}}}
\newcommand{\Rmagc}{\ensuremath{\mathrm{Rm}_\mathrm{crit}}}
\newcommand{\Prandtl}{\ensuremath{\mathrm{Pm}}}
\newcommand{\Peclet}{\ensuremath{\mathrm{Pe}}}
\newcommand{\Mach}{\ensuremath{\mathrm{Ma}}}
\newcommand{\Stiffness}{\ensuremath{\mathrm{S}}}
\newcommand{\Lund}{\ensuremath{\mathrm{S}}}
\newcommand{\Lundc}{\ensuremath{\mathrm{S}_\mathrm{crit}}}
\newcommand{\yt}{\texttt{yt}}
\newcommand{\enzo}{\texttt{Enzo}}
\newcommand{\nosection}[1]{%
  \refstepcounter{section}%
  \addcontentsline{toc}{section}{\protect\numberline{\thesection}#1}%
  \markright{#1}}
\newcommand{\nosubsection}[1]{%
  \refstepcounter{subsection}%
  \addcontentsline{toc}{subsection}{\protect\numberline{\thesubsection}#1}%
  \markright{#1}}


%%%%%%end preamble

\begin{document}
\thispagestyle{empty}
\parindent=0cm
\section*{~}
\vspace{-1.5cm}
\begin{center}
\huge{Evan Anders}\\
\large{Biographical Sketch}\\
\small{Department of Astrophysical and Planetary Sciences\\
University of Colorado, Boulder \\\texttt{evan.anders@colorado.edu}}
\end{center}

\vspace{-1cm}
\section*{Professional Preparation}
\vspace{-0.25cm}
\begin{tabular}{lll}
  Whitworth University & Physics & BS, May 2014\\
		       & Math \& Computer Science & Minors, May 2014\\
  \\
  University of Colorado, Boulder & Astrophysics & PhD, Expected Graduation: 2019\\
\end{tabular}
\vspace{-0.5cm}
\section*{Most Relevant Courses}
\vspace{-0.25cm}
\begin{tabular}{llll}
  Fluid Mechanics & Plasma Physics &  Computational Physics  & Software Engineering\\
\end{tabular}
\vspace{-0.5cm}
\section*{Relevant Research Experience}
\vspace{-0.25cm}
\begin{itemize}
\setlength{\itemsep}{-\parsep}
\setlength{\topsep}{-2\parsep}
\setlength{\partopsep}{-2\parsep}

\item LIGO (Laser Interferometer Gravitational-wave Observatory), Hanford Observatory.
NSF SURF Fellow, Summer 2013. Project: Spectral Line Monitoring Tool.

\item Pacific Northwest National Laboratory (PNNL), DOE SULI Intern, Summer 2012. Project: Global Arrays in NumPy (GAiN).


\end{itemize}

\vspace{-1cm}

\section*{Summary}
\vspace{-0.25cm}
Anders has a strong background in physics and computational methods.
He learned techniques used in the development of professional software while
working as a team to develop an Android application for his undergraduate
school newspaper.
Thanks to his time at PNNL, he has developed and improved parallel algorithms
and learned the difficulties intrinsic to large-scale computation.  During his
time at LIGO Hanford Observatory, he dealt with large data sets and learned
computational techniques for organization, storage, and visualization.
His coursework has included numerous large computational projects, including an
undergraduate simulation of interacting charged particles and a graduate level
implementation of Maxwell's equations. 

\end{document}
